\documentclass{homework}

\title{Assignment4: Information Value and Markov Basics}
\author{
  Dmitrii, Maksimov\\
  \texttt{dmitrii.maksimov@fau.de} \\
  \texttt{ko65beyp}
  \and
  Ilia, Dudnik\\
  \texttt{ilia.dudnik@fau.de}\\
  \texttt{ex69ahum}
  \and
  Aleksandr, Korneev\\
  \texttt{aleksandr.korneev@fau.de}\\
  \texttt{uw44ylyz}
}
\begin{document}

\maketitle

\exercise[4.4 (Prediction, Filtering, Smoothing)]
Consider an HMM consisting of a stationary Markov process and a stationary sensor model. We restrict attention to HMMs with a single state-variable $X_t$ and a single evidence variable $E_t$. (The sensor model does not necessarily have the Markov property.)
Explain the results of the prediction, filtering, smoothing algorithms. For each one, state the motivation, the expression for the conditional probability that is to be computed, and explain the components of the formula. You do not have to explain how the algorithms work.
\begin{itemize}
	\item Prediction

	Prediction is the estimation of future state $X_{t+k}, k>0$ given observed evidences $E_{1:t}: P(X_{t+k}|E_{1:t})$.
	\item Filtering

	Filtering is the estimation of an unobserved state $X_t$ given observed evidences $E_{1:t}: P(X_t|E_{1:t})$.

	\item Smoothing

	Smoothing is the retroactive estimation of an unobserved state $X_k, 0\leq k<t$ given observed evidences $E_{1:t}: P(X_k|E_{1:t})$.
\end{itemize}
\end{document}