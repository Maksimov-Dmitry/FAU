\documentclass{homework}
\title{Assignment1: Probability}
\author{
  Dmitrii, Maksimov\\
  \texttt{dmitrii.maksimov@fau.de} \\
  \texttt{ko65beyp}
  \and
  Ilia, Dudnik\\
  \texttt{ilia.dudnik@fau.de}\\
  \texttt{ex69ahum}
  \and
  Aleksandr, Korneev\\
  \texttt{aleksandr.korneev@fau.de}\\
  \texttt{uw44ylyz}
}
\begin{document}

\maketitle

\exercise[1.3 (Basic Probability)]
Which of the following equalities are always true?
\begin{enumerate}
	\item $P(b)=P(a,b)+P(\lnot a, b)$
	
	Always true.  It's the sum rule(i.e. marginalization): $a$ and $\lnot a$ all events for the $A$ variable.
	\item $P(a) = P(a|b) + P(a|\lnot b)$
	
	Never true.  $P(a) = P(a|b)\cdot P(b) + P(a|\lnot b)\cdot P(\lnot b)$.  
	\newline Let's say that $A=a$ - first ball is red and $B=b$ - second ball is blue, where $A,B\in \{a,b\}$. Given equation above we can conclude that $P(A=a) = 1$, which is wrong.
	\item $P(a,b)=P(a)\cdot P(b)$
	
	Not always true. Correct if $A$ and $B$ are independent.
	\item $P(a,b|c)\cdot P(c)=P(c,a|b)\cdot P(b)$
	
	Always true.  From both sides of the equation are $P(a,b,c)$.
	\item $P(a\lor b)=P(a)+P(b)$

	Not always true. Correct if $A$ and $B$ are disjoint.
	\item $P(a,\lnot b)=(1-P(b|a))\cdot P(a)$

	Always true. Product rule, where $P(\lnot b|a)=1-P(b|a)$
\end{enumerate}

\end{document}