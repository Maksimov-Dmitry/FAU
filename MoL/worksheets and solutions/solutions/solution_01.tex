\documentclass{homework}
\usepackage{xcolor}
\usepackage{nicematrix}
\usepackage{booktabs}
\usepackage{enumitem}
\usepackage{caption}
\usepackage{subcaption}
\NiceMatrixOptions{cell-space-limits = 1pt}

\title{Solutions worksheet 01}
\author{
  Maksimov, Dmitrii\\
  \texttt{dmitrii.maksimov@fau.de}
  \and
  Ilia, Dudnik\\
  \texttt{ilia.dudnik@fau.de}
}

\begin{document}

\maketitle

\exercise[{[Solving linear equation systems]}]
Solve the linear equation system $Ax=b$. \emph{A} and \emph{b} given as

$$
A = \begin{pNiceMatrix}
1 & 2 & 3\\
1 & 4 & 9\\
1 & 8 & 27
\end{pNiceMatrix}
, b= \begin{pNiceMatrix}
3\\
2\\
7
\end{pNiceMatrix}
$$

Gauss elimination method:

\begin{enumerate}
	\item Augmented matrix
	\[
	\begin{bNiceArray}{rrrcr}
	        1 & 2 & 3   &|& 3 \\
	        1 & 4 & 9   &|& 2 \\
	        1 & 8 & 27 &|& 7
	\end{bNiceArray}
	\]
	\item Add -1 times row (1) to both row (2) and row (3)
	\[
	\begin{bNiceArray}{rrrcr}[last-col]
	      1 & 2 & 3   &|& 3  & \\
	      0 & 2 & 6   &|& -1 & R_2 - R_1 \\
	      0 & 6 & 24 &|& 4  & R_3 - R_1
	\end{bNiceArray}
	\]
	\item Divide all terms in row (2) by 2 and add -3 times row (2) to row (3)
	 \[
	\begin{bNiceArray}{rrrcr}[last-col]
	      1 & 2 & 3 &|& 3 & \\
	      0 & 1 & 3 &|& -\frac{1}{2} & \frac{R_2}{2} \\
	      0 & 0 & 6 &|& 7 & R_3 - 3R_2
	\end{bNiceArray}
	 \]
	\item Divide all terms in row (3) by 6
	 \[
	\begin{bNiceArray}{rrrcr}[last-col]
	      1 & 2 & 3 &|& 3 & \\
	      0 & 1 & 3 &|& -\frac{1}{2} & \\
	      0 & 0 & 1 &|& \frac{7}{6} & \frac{R_3}{6}
	\end{bNiceArray}
	 \]
	\item Row echelon form

	Add -3 times row (3) to both row (2) and row (1)
	 \[
	\begin{bNiceArray}{rrrcr}[last-col]
	      1 & 2 & 0 &|& -\frac{1}{2} & R_1 - 3R_3\\
	      0 & 1 & 0 &|& -4 & R_2 - 3R_3 \\
	      0 & 0 & 1 &|& \frac{7}{6} &
	\end{bNiceArray}
	 \]

	Add -2 times row (2) to row (1)
	 \[
	\begin{bNiceArray}{rrrcr}[last-col]
	      1 & 0 & 0 &|& \frac{15}{2} & R_1 - 2R_2\\
	      0 & 1 & 0 &|& -4 &  \\
	      0 & 0 & 1 &|& \frac{7}{6} &
	\end{bNiceArray}
	 \]
\end{enumerate}

Answer: 
$$
x = \begin{pNiceMatrix}
\frac{15}{2}\\
-4\\
\frac{7}{6}
\end{pNiceMatrix}
$$

\exercise[{[Norms]}]
\begin{definition*}[Norm]
	A mapping $\norm{\cdot}$ from any (real) vector space \emph{V} to the real numbers $\R$ is called a norm, whenever
	\begin{enumerate}
		\item $\norm{v+w} \leq \norm{v} + \norm{w}$
		\item $\norm{v} = 0 \Longrightarrow v = 0_V$
		\item $\norm{\lambda v} = |\lambda|\cdot \norm{v}$
	\end{enumerate}
	for all $\lambda \in \R, v,w \in V$
\end{definition*}
\begin{enumerate}
	\item $\text{Let} V = \R^n \text{ for some } n \in \N.$ The euclidean norm
	$$\norm{v}_2 \coloneqq \sqrt{\sum_{i=1}^{n} v_i^2}$$
	is a norm.

	Proof:
	\begin{enumerate}
		\item Proof of 1 statement
		\[\norm{v+w}_2^2 \coloneqq \sum_{i=1}^{n} (v_i+w_i)^2 = \sum_{i=1}^{n} v_i^2+2 v_i w_i + w_i^2=
		\sum_{i=1}^{n} v_i^2+\sum_{i=1}^{n} 2 v_i w_i + \sum_{i=1}^{n} w_i^2 = \norm{v}_2^2 + 2(v\cdot w) + \norm{w}_2^2\]
		Taking into account the Cauchy-Schwarz Inequality
		$$|v\cdot w| \leq \norm{v}_2 * \norm{w}_2$$
		which implies
		$$\norm{v}_2^2 + 2(v\cdot w) + \norm{w}_2^2\ \leq \norm{v}_2^2 + 2\norm{v}\norm{w} + \norm{w}_2^2 = (\norm{v} + \norm{w}) ^ 2$$
		Hence, 
		$$\norm{v+w}_2^2 \leq  (\norm{v}_2 + \norm{w}_2) ^ 2$$
		$$\norm{v+w}_2 \leq \norm{v}_2 + \norm{w}_2$$
		as required
		\item Proof of 2 statement
		\[\norm{v}_2 \coloneqq \sqrt{\sum_{i=1}^{n} (v_i)^2}\]
		\[\sqrt{\sum_{i=1}^{n} v_i^2} = 0 \Longleftrightarrow v_i = 0, \: \forall i \]
		which implies
		$$v = 0_V$$
		as required
		\item Proof of 3 statement
		\[\norm{\lambda v}_2 \coloneqq \sqrt{\sum_{i=1}^{n} (\lambda v_i)^2} = |\lambda|\cdot \sqrt{\sum_{i=1}^{n} v_i^2} = |\lambda|\cdot \norm{v}\]
		as required
	\end{enumerate}
	Hence, the euclidean norm is a norm.
	\item $\text{Let} V = \R^n \text{ for some } n \in \N.$ The mapping
	\[\norm{v}_{\frac{1}{2}} \coloneqq (\sum_{i=1}^{n} \sqrt{|v_i|})^2\]
	is a norm.

	Proof: let $v = (0,1), w = (1,0)$, so $v,w \in V.$ Given that
	\[\norm{v + w}_{\frac{1}{2}} \coloneqq (\sum_{i=1}^{2} \sqrt{|v_i + w_i|})^2 = 2^2 = 4\]
	\[\norm{v}_{\frac{1}{2}} + \norm{w}_{\frac{1}{2}} \coloneqq (\sum_{i=1}^{2} \sqrt{|v_i|})^2 + (\sum_{i=1}^{2} \sqrt{|w_i|})^2 = 1 + 1 = 2\]
	\[\norm{v + w}_{\frac{1}{2}} > \norm{v}_{\frac{1}{2}} + \norm{w}_{\frac{1}{2}}\]
	Hence, the $\norm{\cdot}_{\frac{1}{2}}$ is not a norm.
	\item Let \emph{V} be the space of convergent sequences. The mapping
	\[\norm{v}_{lim} \coloneqq \lim_{n\to\infty} v_n\]
	is a norm.

	Proof: 
	\[\norm{\lambda \cdot v}_{lim} \coloneqq \lim_{n\to\infty} \lambda \cdot v_n = \lambda \cdot \lim_{n\to\infty} v_n = \lambda \cdot \norm{v}_{lim} 
	\neq |\lambda| \cdot \norm{v}_{lim}\]
	Hence, the $ \norm{\cdot}_{lim}$ is not a norm.
\end{enumerate}

\exercise[{[Python, Pandas, K-Means]}]
\begin{enumerate}[label=(\alph*)]
	\item DS analysis
		
		Let's start by describing DS

		\begin{tabular}{lrr}
\toprule
{} &   eruptions &     waiting \\
\midrule
count &  272.000000 &  272.000000 \\
mean  &    3.487783 &   70.897059 \\
std   &    1.141371 &   13.594974 \\
min   &    1.600000 &   43.000000 \\
25\%   &    2.162750 &   58.000000 \\
50\%   &    4.000000 &   76.000000 \\
75\%   &    4.454250 &   82.000000 \\
max   &    5.100000 &   96.000000 \\
\bottomrule
\end{tabular}

		\begin{figure}[h]
			\centering
			\includegraphics[width=0.8\textwidth]{df_visualization.png}
			\caption{visualization of \emph{faithful.csv} data}
		\end{figure}
	\item KMeans clustering visualization

		\begin{figure}
		     \centering
		     \begin{subfigure}[b]{0.6\textwidth}
		         \centering
		         \includegraphics[width=\textwidth]{kmeans_visualization_0.png}
		         \caption{iteration = 0}
		     \end{subfigure}
		     \hfill
		     \begin{subfigure}[b]{0.6\textwidth}
		         \centering
		         \includegraphics[width=\textwidth]{kmeans_visualization_1.png}
		         \caption{iteration = 1}
		     \end{subfigure}
		     \hfill
		     \begin{subfigure}[b]{0.6\textwidth}
		         \centering
		         \includegraphics[width=\textwidth]{kmeans_visualization_2.png}
		         \caption{iteration = 2}
		     \end{subfigure}
		     \hfill
		     \begin{subfigure}[b]{0.6\textwidth}
		         \centering
		         \includegraphics[width=\textwidth]{kmeans_visualization_3.png}
		         \caption{iteration = 3}
		     \end{subfigure}
		        \caption{KMeans algorithm}
		\end{figure}

		\begin{figure}
			\centering
			\includegraphics[width=0.8\textwidth]{kmeans_overall.png}
			\caption{KMean clustering sumup}
		\end{figure}
\end{enumerate}
\newpage
\exercise*[{[Implementing EM for Clustering]}]

\end{document}