\documentclass{homework}
\usepackage{xcolor}
\usepackage{nicematrix}
\usepackage{booktabs}
\usepackage{enumitem}
\usepackage{caption}
\usepackage{subcaption}
\usepackage{leftidx}

\NiceMatrixOptions{cell-space-limits = 1pt}

\title{Solutions 01 BayesianClassifier}
\author{
  Maksimov, Dmitrii\\
  \texttt{dmitrii.maksimov@fau.de}
  \and
  Ilia, Dudnik\\
  \texttt{ilia.dudnik@fau.de}
  \and
  Isa, Baghirov\\
  \texttt{isa.baghirov@fau.de}
  \and
  Yulia, Herasymenko\\
  \texttt{yuliia.herasymenko@fau.de}
}

\begin{document}

\maketitle

\exercise
A computer science student is annoyed that the two thirds of the e-mails he/she
receives is spam. Therefore, he/she decides to write a classifier that should decide whether an incoming e-mail is spam (class y=1) or ham (class y=0). For
classification the Bayes classifier is used. The student notices that in spam and
ham mails, certain words occur with different probability. Therefore, the student
bases the classification on the words \textbf{\emph{x}} = \{\emph{Viagra, bet, student, sports, cinema}\}. By inspecting all his/her previous mails, the student estimates that in the ham mails, the probabilities are 0\% for Viagra, 10\% for bet, 40\% for student, 30\% for
sports, and 10\% for cinema. In the spam mails, the words Viagra occurs in 50\%,
bet in 30\%, student in 5\%, and sports and cinema in 2\% of the mails.

The student does not count how often each word occurs, but only whether it is present in the mail. For simplicity, he/she assumes that the words occur independently.

\begin{enumerate}[label=(\alph*)]
\item Has student consider all the postulates of pattern recognition?

	For this example 4 postulates shoul be considered:
	\begin{enumerate}[label=\arabic*]
	\item Sample

	There are two classes for student's emails and classes have their own features. Their e-mails are tha data $\triangleq \Omega$ and there are data for each 2 classes.
	\[\omega = \{\leftidx{^1}{f(x)}{}, \leftidx{^2}{f(x)}{}\} \in \Omega \]
	\item Features

	Since the words $\in \textbf{x}$ occur with different probability, features can be used to distinguish one class from another.
	\item Compactness

	As our features are vectors we can measure distance between them. And considering the fact that some words are likely to occur in specific class distance should be small between members of one class and large for different ones.
	\item Similarity
	
	A distance measure shoud be chosen to check whether patterns be similar. 
	\end{enumerate}
\item Write down the priors for an e-mail being spam or ham.

$p(y=1) = \frac{2}{3}, p(y=0) = \frac{1}{3}$
\item Write down the class-conditional probabilities $p(\textbf{x}|y=0)$ and $p(\textbf{x}|y=1)$ for an arbitrary feature vector \textbf{\emph{x}}.

 $\text{Let } \textbf{x} \in \R^n, y \in \{0, 1\} \text{ and } x_i \sim Bernoulli(p)$, then
\[p(\textbf{x}|y = j) = \prod_{i=1}^n p(x_i|y=j) = \prod_{i=1}^n p(x_i = 1|y=j)^{x_i}\cdot (1-p(x_i = 1|y=j)^{1 - x_i}, \text{for } j \in \{0, 1\}\]

\item Write down the Bayesian decision rule for the spam classification problem. Classify the following e-mail using the decision rule:

\emph{Hi, As we talked about yesterday, I want to make a bet with you about the upcoming soccer match. I clearly know more about sports than you. I bet 5\$ against N{\"u}rnberg.}

Decision rule
\begin{equation*}
	f(\textbf{x}_{email})=
	\begin{cases} 
		1, &  p(y = 1|\textbf{x}_{email})\geq p(y = 0|\textbf{x}_{email}), \\ 
		0, & \text{otherwise}.
	\end{cases}
\end{equation*}

$\{\text{e-mail}\} \cap \textbf{\emph{x}} = \{bet, sports\} \triangleq \textbf{x}_{email}$
\[p(\textbf{x}_{email}|y=0) = p(bet|y = 0) \cdot p(sports|y = 0) = 0.1 \cdot 0.3 = 0.03\]
\[p(\textbf{x}_{email}|y=1) = p(bet|y = 1) \cdot p(sports|y = 1) = 0.3 \cdot 0.02 = 0.006\]
\[p(y = 0|\textbf{x}_{email}) =p(y=0) \cdot p(\textbf{x}_{email}|y=0)) = \frac{1}{3} * 0.03 = 0.01\]
\[p(y = 1|\textbf{x}_{email}) =p(y=1) \cdot p(\textbf{x}_{email}|y=1)) = \frac{2}{3} * 0.006 = 0.004\]
\[f(\textbf{x}_{email}) = 0\]
	
\end{enumerate}

\end{document}