\documentclass{homework}
\usepackage{xcolor}
\usepackage{nicematrix}
\usepackage{booktabs}
\usepackage{enumitem}
\usepackage{caption}
\usepackage{subcaption}
\usepackage{leftidx}
\usepackage{mathrsfs}
\usepackage{pgfplots, pgfplotstable}

\NiceMatrixOptions{cell-space-limits = 1pt}

\title{Simulation and Modeling I, Assignment 2}
\author{
  Dmitrii, Maksimov\\
  \texttt{dmitrii.maksimov@fau.de}
  \and
  Priyanka, Muddarla\\
  \texttt{priyanka.muddarla@fau.de}
  \and
  Susmitha, Palla\\
  \texttt{susmitha.palla@fau.de}
  \and
  Bhaskar Jyoti, Gogoi\\
  \texttt{bhaskar.j.gogoi@fau.de}
}


\begin{document}

\maketitle

\exercise
In unsorted sets with n different items, we perform sequential and random searches for one particular item (of which we know that it is included exactly once in each set).

\emph{Sequential search} means that we pick one item after the other from the set until the desired item is found.

\emph{Random search} means that we check an arbitrary item from the set, but leave it in the set, and recheck until the desired item is found.

\begin{enumerate}
	\item What is the \textbf{mean number of checks} (including accidental rechecks in random search) until the item is found in a set \textbf{in either case} (with sequential and with random search)? Which \textbf{distribution} does the random variable “number of checks” have \textbf{in each case}? Simply give the names and characteristic parameters of these distributions.

	In either cases the distributions should be the probability distribution of the number \emph{X} of trials needed to get one success, supported on the set \{1, 2, \dots, n\}.	
	\begin{itemize}
		\item sequential search
		
		There is no difference in having success exactly on the first check or on the 10th check. Hence, this is the \emph{discrete uniform distribution} with $a = 1$ and $b = n$, $X \in \{1, 2, ..., n\}$. The probability density function in this case: \[f_X(X = x|n) = \frac{1}{n}.\]
		The mean of the uniform distribution: $\frac{a+b}{2}$. Hence, \[E[X] = \frac{n+1}{2}.\]
		\item random search

		This is \emph{geometric distribution} with $p = \frac{1}{n}$ and $X \in \R$. Given that, \[f_X(X = x|n) = \frac{1}{n}\cdot (1 - \frac{1}{n}) ^ {x - 1}.\]
		The mean of the geometric distribution: $\frac{1}{p}$. Hence, \[E[X] = n.\]
	\end{itemize}
	\item For the distribution arising from random search: show \textbf{formally} that this distribution enjoys the \textbf{memoryless property} (in analogy to the bottom of slide 33 of the exercise video “Basics of Probability Theory”)! Interpret this memoryless property for random search \textbf{informally} (i.e., in your own words).
	\begin{itemize}
		\item formally

		If $f_X(x)$ for the random search is memoryless, then $P(X > x + z | X > x) = P(X > z)$. Hence, 
		\begin{align*}
		P(X > x + z | X > x) &= \frac{P(X > x + z)}{P(X > x)} \\
		&= \frac{(1-\frac{1}{n})^{x + z}}{(1 - \frac{1}{n})^{x}} \\
		&= (1-\frac{1}{n})^z \\
		&= P(X>z)
		\end{align*}
		\item informally
		
		Let us imagine that we have already checked 10 times and still have not found an item. So, we want to know the probability that after 5 more attempts we still will not find an item. In this case, the information about previous 10 unsuccessful attempts does not affect the outcome of the next 5 ones.
	\end{itemize}
\end{enumerate}

\exercise*
In technical systems and software programs, processes are often started at the same time and run in parallel. Before the next phase of the system/program can be entered, sometimes all of these processes have to be finished or sometimes simply a single one. We define the runtime of process \textbf{k} with the random variable $X_k$.

Let $X_1, X_2,\dots,X_n$ be (mutually) independent and let $Y = \textbf{min}\{X_1, X_2,\dots,X_n\}$ and \newline $Z = \textbf{max}\{X_1, X_2,\dots,X_n\}$. What do \emph{Y} and \emph{Z} model?

\begin{itemize}
	\item $Y$ models the start time of next phase in case of completion at least one process
	\item $Z$ models the start time of next phase in case of completion all processes
\end{itemize}

\begin{enumerate}[label=(\alph*)]
	\item Show that the distribution functions $F_Y$ of $Y$ and $F_Z$ of $Z$ are given by \newline $F_Y(t) = 1 - \prod_{i=1}^{n} (1 - F_{X_i}(t))$ and $F_Z(t) = \prod_{i=1}^{n} F_{X_i}(t)$, respectively.
	\begin{itemize}
		\item According to $Z$ description: $P(Z\leq t) = P(X_1\leq t \bigcap X_2\leq t \bigcap \dots \bigcap X_n\leq t)$. Since $X_1, X_2,\dots,X_n$ are mutually independent, then \newline $P(Z\leq t) = P(X_1\leq t)\cdot P(X_2\leq t) \cdot \dots \cdot P(X_n\leq t) \Rightarrow F_Z(t) = \prod_{i=1}^{n} F_{X_i}(t)$.
		\item Let us find the probability that all processes are still running: $\prod_{i=1}^{n} (1 - F_{X_i}(t))$, the proof of this is the same as one above. Hence, Y probability will be 1 - this probability $\Rightarrow F_Y(t) = 1 - \prod_{i=1}^{n} (1 - F_{X_i}(t))$
	\end{itemize}
	\item Specialize (and simplify) these distribution functions for \textbf{independent and identically exponentially distributed} $X_i (i=1,\dots,n)$!
What distribution do you get for \textbf{Y}? What is the value of \textbf{E[Y]} in this case (exponentially distributed $X_i$)? Describe this observation in your own words.

	Given $F_{X_i}(t) = 1 - e^{-\lambda t}, \text{ where } \lambda \in \R_{>0}$:
	\begin{itemize}
		\item $F_Z(t) = (1 - e^{-\lambda t})^n$
		\item $F_Y(t) = 1 - (1 - (1 - e^{-\lambda t}))^n = 1 - e^{-\lambda n t}.\, E[Y] = \int_0^\infty t\lambda n\cdot e^{-\lambda n t})\,dt = \frac{1}{n\lambda}$.
			\newline It is obvious that the more processes the less the average time of Y. Hence, the $E[Y]$ is inversely proportional to $n$.
	\end{itemize}
\end{enumerate}
\exercise*
A systems runs by a generator that provides power for the system. If the generator fails, the system has a battery, which can supply it with power \textbf{for exactly five more days}. Let the time to failure X for the generator be exponentially distributed with \textbf{expectation 1700} days, while Y denotes the time to the complete failure of the system.
\begin{enumerate}[label=(\alph*)]
	\item What is the \textbf{probability} that the generator fails \textbf{within the first 1700 day}? What is the quantile with respect to this probability value \textbf{for random variable Y}?

	The expected value of an exponentially distributed random variable X: $\frac{1}{\lambda} \Rightarrow \lambda = \frac{1}{1700}$. $F_X(X\leq 1700) = 1 - e^{-\frac{1700}{1700}} \approx 0.63$. $F_Y(t) = 1 - e^{-\lambda(y-5)} \Rightarrow F^{-1}_Y(F_X(X\leq 1700)) = \newline -\ln(1 - F_X(X\leq 1700))\cdot 1700 + 5 = 1705$

	\item Compute the \textbf{coefficient of variation for both X and Y}!
		\begin{itemize}
			\item $C_X = \frac{\sigma_X}{E[X]} = \frac{\sqrt{Var[X]}}{E[X]} = 1$
			\item $C_Y = \frac{\sigma_Y}{E[Y]} = \frac{\sqrt{Var[Y]}}{E[Y]} = \frac{1700}{1705}$
		\end{itemize}
	\item What is the (coefficient of) \textbf{correlation between X and Y}?
	\begin{align*}
		p_{X,Y} &= \frac{C_{X,Y}}{\sigma_X \sigma_Y} \\
		&= \lambda^2(E[XY] - E[X]E[Y]) = 1
	\end{align*}
	Honestly, we do not know how to calculate $E[XY]$. This is because, in our point of view, $E[XY] = \int_0^\infty \int_{x+5}^{x+5} x\cdot y \cdot \lambda e^{-\lambda x} \cdot \lambda e^{-\lambda (y-5)}\,dydx = 0$, but in this case $p_{X,Y} \neq 1$. Since $Y = X + 5$, it is obvious that $p_{X,Y} = 1$.
\end{enumerate}

\exercise*
Assume a single-server queue with one service unit (one packet can be processed at each time). For this single-server queue, the interarrival times of packets are exponentially distributed (with rate $\lambda$) and the service time for serving one packet is exponentially distributed (with rate $\mu$). This is also called M/M/1 queue. The mean number of customers in this system (including the service unit) is computed as:
\[N = \frac{U}{1 - U}, \text{ where } U = \frac{\lambda}{\mu}.\]
Besides, if we assume that the service time is generally distributed (arbitrary / any distribution), then the queue is called M/G/1 queues. We assume that the service times have the same mean as above ($\frac{1}{\mu}$). Then the mean waiting time (i.e., mean delay in the queue excluding the service unit) is given by: \[W = \frac{U(c_s^2 + 1)}{2\mu(1-U)}\] where $c_s$ is the coefficient of variation of the service time distribution.
\begin{enumerate}[label=(\alph*)]
\item Under which condition (on the arrival rate $\lambda$) do the mean delay (D) in the system (including the service unit) and the mean number (N) of customers in the system have the same numeric values for an M/M/1 queue?

Let $S$ - the service time, then $D = W + E[S] = \frac{U(c_s^2 + 1)}{2\mu(1-U)} + \frac{1}{\mu}$. \newline Since, $c_s = 1$: $D = \frac{U(1 + 1)}{2\mu(1-U)} + \frac{1}{\mu} =  \frac{\lambda}{\mu(\mu - \lambda)} + \frac{1}{\mu}$.

$N = \frac{U}{1 - U} = \frac{\lambda}{\mu - \lambda}$. Pooling everything together: $\frac{\lambda}{\mu(\mu - \lambda)} + \frac{1}{\mu} = \frac{\lambda}{\mu - \lambda} \Rightarrow \lambda = 1$.
\item For identical mean service times, by which factor do the mean waiting times $W^{M/M/1}$ and $W^{M/D/1}$ differ for an M/M/1 and an M/D/1 queue? (D stands for deterministic / constant service time).

$W^{M/M/1} = \frac{U(1 + 1)}{2\mu(1-U)} = \frac{U}{\mu(1-U)}$.
\newline Since $c_{const} = 0$, $W^{M/D/1} = \frac{U(0 + 1)}{2\mu(1-U)} = \frac{U}{2\mu(1-U)}$.
\newline Hence, $W^{M/M/1} = 2W^{M/D/1}$.

\end{enumerate}


\end{document}